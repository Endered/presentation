\documentclass{beamer}

\usepackage{luatexja-fontspec}

\usepackage{tcolorbox}

\usetheme{default}
\usecolortheme{dove}

\begin{document}

\title{X Window Systemで遊んだ}
\subtitle{〜嗚呼、素晴らしきかなX Window System〜}
\author{Endered}
\date{2023年06月23日}

\begin{frame}
  \titlepage
\end{frame}


\begin{frame}
  \frametitle{自己紹介}
  \begin{itemize}
  \item 名前: Endered
  \item 所属: 会津大学大学院1年
  \item 趣味: Lisp / 言語処理計
  \item 好きなもの: 関数型言語 / \alert<2>{Linux} \onslide<2>{ $\leftarrow$ なぜ好きなのか?}
  \item その他: スライドを \LaTeX で書いてみました(楽しかった)
  \end{itemize}
\end{frame}

\begin{frame}[c]{}
  \frametitle{なぜLinuxが好きなのか}
  \centering
  色々理由はありますが、特筆するとしたら

  \onslide<2>{X Window Systemがあるから}
\end{frame}

\begin{frame}
  \frametitle{X Window Systemとは}
  GUIに関する機能を提供するための\onslide*<1>{通信規格}\onslide*<2>{\alert{通信規格}} \& その実装
  \begin{itemize}
  \item 提供している機能
    \begin{itemize}
    \item 画面にウィンドウを描画
    \item マウスを動かす \& クリックの処理
    \item キーボード入力を処理
    \end{itemize}
  \item<2> 通信規格で定義されているので非常に便利
  \end{itemize}
\end{frame}

\begin{frame}
  \frametitle{すごいぞX Window System(X転送について)}
  X転送とは
  \begin{itemize}
  \item 我々の感覚としては画面転送(VNC)が近いです
  \item ただし、描画命令をネットワーク越しに送っている
  \end{itemize}
\end{frame}

\begin{frame}
  \frametitle{実際にX転送を使ってみよう}

  以下のコマンドを実行するだけ

  \begin{tcolorbox}[title=emacsをローカルに転送する例]
    \texttt{ssh -Y "適当なサーバ" emacs}
  \end{tcolorbox}
\end{frame}

\begin{frame}
  \frametitle{X転送の何がすごいのか}
  \begin{itemize}
  \item X転送はX Window Systemの\textbf{描画命令}を転送している
    \begin{itemize}
    \item VNCなどのソフトウェアは描画結果を転送している
    \end{itemize}
  \item<2-> つまり、X転送では(理論上)FPSが遊べる
    \begin{itemize}
    \item VNCだと: マウスカーソルが上手に処理できなくて画面が吹っ飛ぶ
      \begin{itemize}
      \item<3-> マウスカーソルの位置を転送しているから
      \end{itemize}
    \item X転送だと: 普通に遊べる(GPUで描画する系は遅い)
      \begin{itemize}
      \item<3-> マウスカーソルの移動イベントを転送しているから
      \end{itemize}
    \end{itemize}
  \item<4> リモートで起動したソフトウェアをローカルで動いているかのように使える
  \end{itemize}
\end{frame}

\begin{frame}
  \frametitle{ssh -Y "サーバ名"がやっていること}
  \begin{tcolorbox}[title=注意]
    \begin{itemize}
    \item ここではリモートの\texttt{\$DISPLAY}を$x$、ローカルの\texttt{\$DISPLAY}を$y$だとします。
    \item また、環境のよっては動かない場合があります。
    \end{itemize}
  \end{tcolorbox}
  \begin{enumerate}
  \item<2-> リモートのポート $6000 + x$ をローカルの $6000 + y$ にフォワーディングする
  \item<3-> ローカルのディスプレイ$y$へ描画するのに必要な鍵を取得する
  \item<4-> リモートの$x$へ描画する際に先程取得した鍵を使うようにする
  \item<5-> 以上(恐らく)
  \end{enumerate}
\end{frame}

\begin{frame}
  \frametitle{X転送を自分でやってみる}
  実践してみましょう
  \begin{itemize}
  \item リモートのポート $6000 + x$ をローカルの $6000 + y$ にフォワーディングする
    \begin{itemize}
    \item<2-> ssh -R 6123:/tmp/.X11-unix/X0
    \item<2-> \textreferencemark ここでは、フォワーディング先をポートではなくソケットにしている
    \end{itemize}
  \item ローカルのディスプレイ$y$へ描画するのに必要な鍵を取得する
    \begin{itemize}
    \item<3-> \texttt{xauth list \$DISPLAY | grep -oP '\textbackslash w*\$'}
    \end{itemize}
  \item リモートの$x$へ描画する際に先程取得した鍵を使うようにする
    \begin{itemize}
    \item<4-> \texttt{xauth add \$DISPLAY . "先程取得した鍵"}
    \item <4-> この際、\texttt{\$DISPLAY}を\texttt{:123}に設定する
    \end{itemize}
  \end{itemize}
\end{frame}

\begin{frame}[c]{}
  \centering
  動いた〜
\end{frame}

\begin{frame}
  \frametitle{作ったものの紹介}
  X Window Systemを理解するためにちょっとしたソフトウェアを作ってみました

  \pause

  その名も \textit{Space Ship Cursor}
\end{frame}

\begin{frame}
  \frametitle{操作説明}
  ソフトを起動すると宇宙船を模したカーソルが出現するよ
  \begin{itemize}
  \item WSで前進/後進、ADで左右に回転!
  \item Jでマウスクリック!
  \end{itemize}
  \pause
  ということでマウスを使用するゲームでも遊ぼうと思います
\end{frame}

\end{document}
